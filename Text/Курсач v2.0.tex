\documentclass[12pt, a4paper, fleqn]{extreport}

\usepackage{amsthm,amsfonts,amsmath,amssymb,amscd}
\usepackage[T2A]{fontenc}                         
\usepackage[utf8]{inputenc}                      
\usepackage[english, russian]{babel}
\usepackage{graphicx}
\usepackage{indentfirst}
\usepackage{cite} 
\usepackage{psfrag}
\usepackage{subcaption}
\usepackage[top=2cm,bottom=2cm,left=2.5cm,right=1cm]{geometry}

\linespread{1.3} 

\usepackage{ucs}
\usepackage{lineno}
\usepackage{graphicx}

\usepackage{float}

\begin{document}
	
	\newcommand{\round}[1]{\stackrel{o}{#1}}
	
	\tableofcontents
	\clearpage
	
	%%%%%%%%%%%%%%%%%%%%%%%%%%%%%%%%%%%%%%%%%%%%%%%%%%%%%%%%%%%%%%%%%%%%%%%%%%%%%%
	%%%%%%%%%%%%%%%%%%%%%%%%%%%%%%%%%%%%%%%%%%%%%%%%%%%%%%%%%%%%%%%%%%%%%%%%%%%%%%
	\chapter*{Введение}
	%%%%%%%%%%%%%%%%%%%%%%%%%%%%%%%%%%%%%%%%%%%%%%%%%%%%%%%%%%%%%%%%%%%%%%%%%%%%%%
	%%%%%%%%%%%%%%%%%%%%%%%%%%%%%%%%%%%%%%%%%%%%%%%%%%%%%%%%%%%%%%%%%%%%%%%%%%%%%%
	\addcontentsline{toc}{chapter}{Введение} 
	
	
	
	
	%%%%%%%%%%%%%%%%%%%%%%%%%%%%%%%%%%%%%%%%%%%%%%%%%%%%%%%%%%%%%%%%%%%%%%%%%%%%%%
	%%%%%%%%%%%%%%%%%%%%%%%%%%%%%%%%%%%%%%%%%%%%%%%%%%%%%%%%%%%%%%%%%%%%%%%%%%%%%%
	\chapter{Разностные схемы для уравений Эйлера в сферической системе координат.}
	%%%%%%%%%%%%%%%%%%%%%%%%%%%%%%%%%%%%%%%%%%%%%%%%%%%%%%%%%%%%%%%%%%%%%%%%%%%%%%
	%%%%%%%%%%%%%%%%%%%%%%%%%%%%%%%%%%%%%%%%%%%%%%%%%%%%%%%%%%%%%%%%%%%%%%%%%%%%%%
	
	%%%%%%%%%%%%%%%%%%%%%%%%%%%%%%%%%%%%%%%%%%%%%%%%%%%%%%%%%%%%%%%%%%%%%%%%%%%%%%
	\section{Аппроксимация уравнения неразрывности.}
	%%%%%%%%%%%%%%%%%%%%%%%%%%%%%%%%%%%%%%%%%%%%%%%%%%%%%%%%%%%%%%%%%%%%%%%%%%%%%%
	
	Уравнение неразрывности в сферических координатах
	$x = r\sin\theta\cos\phi, y = r\sin\theta\sin\phi, z = r\cos\theta$
	может быть записано в следующем виде:
	\begin{equation*}
	\begin{split}
		\dfrac{\partial \rho}{\partial t}
			= \dfrac{1}{r^2}\dfrac{\partial}{\partial r}\bigg(r^2\rho\upsilon_r\bigg)
			+ \dfrac{1}{r\sin\theta}\dfrac{\partial}{\partial\theta}\bigg(\sin\theta\rho\upsilon_\theta\bigg)
			+ \dfrac{1}{r\sin\theta}\dfrac{\partial}{\partial\phi}\bigg(\rho\upsilon_\phi\bigg).
	\end{split}
	\end{equation*}
	
	Рассмотрим сферическую ячейку $\Omega$,
	на которой введем систему ортонормированных функций 
	$\psi^m(r, \theta, \phi), m=0..M$.
	Искомые поля будем искать в виде комбинации этих функций:
	\begin{equation*}
	\begin{split}
		\rho \approx \sum\limits_{m=0}^{M} \rho^m(t)\psi^m(r, \theta, \phi).
	\end{split}
	\end{equation*}
	
\end{document}
